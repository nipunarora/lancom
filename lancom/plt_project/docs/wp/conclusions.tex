
\ section
{
Conclusion}

In this paper we described LinkWidth, a single - end controlled tool to
  measure the installed / bottleneck capacity and the available / un -
  utilized bandwidth.
  We give implementation details of how we extended the two existing
techniques (Recursive Packet Train and Train of Packet Pair),
to employ TCP RST packets sandwiched between TCP SYN packets.In addition,
we show how to use a binary search approach to estimate installed / bottleneck and available / un -
utilized capacity through a single tool.
Our experimental results for both installed and available capacity in various scenarios indicate that LinkWidth gives a good estimate of the
installed and available capacity when compared to well accepted tools like PATHCHAR and IPERF.However,
we cannot accurately measure links that exhibit packet loss or are assymetric which seem to require a two -
ended measurement tool.In lab experiments using CISCO routers,
the RPT packet train method reports very accurate measurements of bottleneck capacity.Small cross - traffic packets
do
    not introduce
    significant change to the end - to - end dispersion of the train.This
    decreases our error due to capacity underestimation or overestimation
    (otherwise prevalent in the older packet pair methods)
    . Moreover, use
      of symmetric links / paths ensures that the perceived value of received
      dispersion is not very different from the correct value.
      When we deploy our tool in the Internet from a single vantage point,
      we have no knowledge of what cross - traffic / link conditions to expect
      at the other end of the path.The only way our variant of Train of
      Packet Pair tries to react to congestion due to cross - traffic, is by
      observing the sending and `` perceived '' reception rate of the train of
      packet pairs.Although not as accurate as IPERF a two - end controlled
      tool, we are able to achieve a good estimate of available
      capacity / bandwidth.To optimize our estimates and to converge faster,
      we can use knowledge of the link cross - traffic and available bandwidth
      to `` tune '' some of our tool.However, given the fact that our tool is
      not using any congestion control, we appear to be more aggressive in
      measuring available capacity and thus get higher values when compared
      with tools that use regular two - end TCP connections to perform the
      same measurements.
