We introduce LinkWidth, a method for estimating capacity and available
  bandwidth using single - end controlled TCP packet probes.To estimate
  capacity, we generate a train of TCP RST packets `` sandwiched ''
  between trains of TCP SYN packets.Capacity is computed from the
  end - to - end packet dispersion of the received TCP RST / ACK packets
  corresponding to the TCP SYN packets going to closed ports.Our
  technique is significantly different from the rest of the packet - pair
  based measurement techniques, such as
{
\em CapProbe,}
{
  \em
    pathchar}
    and
    {
    \em pathrate,}
  because the long packet trains minimize errors due
    to bursty cross - traffic.Additionally, TCP RST packets
  do
    not generate
      additional ICMP replies, thus avoiding cross - traffic due to such
      packets from interfering with our probes.In addition, we use TCP
      packets for all our probes to prevent QoS - related traffic shaping
      (based on packet types)
  from affecting our measurements (eg.CISCO
				     routers by default are known have to very
				   high latency while generating to ICMP TTL
				   expired replies)
    . We extend the
    {
    \it Train of Packet Pairs}
  technique to approximate the
    available link capacity.We use a train of TCP packet pairs with
    variable intra - pair delays and sizes.This is the first attempt to
    implement this technique using single - end TCP probes, tested on a
    range of networks with different bottleneck capacities and cross
    traffic rates.The method we use for measuring from a single point of
    control uses TCP RST packets between a train of TCP SYN packets.The
    idea is quite similar to the technique for measuring the bottleneck
    capacity.We compare our prototype with
  {
  \em pathchirp,}
  {
    \em
      pathload,}
      {
      \em IPERF,}
    which require control of both ends as well as
      another single end controlled technique
    {
    \em abget}
     , and demonstrate
      that in most cases our method gives approximately the same results if
      not better.
