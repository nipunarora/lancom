\section {Some Sample Grammatical Constructs / Snippets of LANCOM}
 \label {sec:grammatical_snippets}

This section presents how the language constructs ``look'' like.
  The language constructs of LANCOM are quite similar to Firmato.
  The grammar for LANCOM would dictate how network entities and firewalling
  policies be represented using a device independent language.
  A key goal of our language is clear separation of firewalling policy
  description and topology description.
  Such a separation allows an administrator to independently describe routing
  and firewalling configurations.
  The next subsection shows exactly how this is achieved. 

\subsection{Some Sample Grammatical Constructs}

In this subsection,
  we present some of the grammatical constructs of LANCOM.As of now,
  we limit the capabilities of the language to simply represent configuration
  in the form of topology and role based representation.
  Later we shall expand the language to include conditional and looping
  constructs. \\ 

\begin{enumerate}
 \item \noindent ${\it statement}$ $\rightarrow$ ${\it expression}$ $({\it statement})^{*}$ ${\it delimiter}$ \\ 
  \item ${\it expression}$ $\rightarrow$ ${\it type\_name}$ ${\it object\_name}$ \\ 
  \item ${\it expression}$ $\rightarrow$ ${\it object\_name}$ ${\it attrib}$ ${\it attrib\_values}$ \\\\ 
  
\item \noindent ${\it attrib}$ $\rightarrow$ ${\tt topology}$ \\ 
\item ${\it attrib}$ $\rightarrow$ ${\tt host\_group}$ \\
 \item ${\it attrib}$ $\rightarrow$ ${\tt host}$ \\ 
\item ${\it attrib}$ $\rightarrow$ ${\tt service\_group}$ \\ 
\item ${\it attrib}$ $\rightarrow$ ${\tt role}$ \\ 
\item ${\it attrib}$ $\rightarrow$ ${\tt policy}$ \\\\ 


\item \noindent ${\it attrib\_values}$ $\rightarrow$ ${\it topology}$ \\ 
\item ${\it attrib\_values}$ $\rightarrow$ ${\it host\_group}$ \\ 
\item ${\it attrib\_values}$ $\rightarrow$ ${\it host}$ \\ 
\item ${\it attrib\_values}$ $\rightarrow$ ${\it service}$ \\ 
\item ${\it attrib\_values}$ $\rightarrow$ ${\it service\_group}$ \\ 
\item ${\it attrib\_values}$ $\rightarrow$ ${\it role}$ \\ 
\item ${\it attrib\_values}$ $\ rightarrow$ ${\it policy}$ \\\\ 

\item \noindent ${\it type}$ $\rightarrow$ ${\tt topology\_type\_t}$ \\ 
\item ${\it type}$ $\rightarrow$ ${\tt host\_type\_t}$ \\ 
\item ${\it type}$ $\rightarrow$ ${\tt host\_group\_type\_t}$ \\ 
\item ${\it type}$ $\rightarrow$ ${\tt service\_group\_type\_t}$ \\ 
\item ${\it type}$ $\rightarrow$ ${\tt role\_type\_t}$ \\ 
\item ${\it type}$ $\rightarrow$ ${\tt policy\_t}$ \\ 
\item ${\it object\_name}$ $\rightarrow$ $[{\tt 0-9,a-z,A-Z}]^{+}$ \\\\ 

\item \noindent ${\it delimiter}$ $\rightarrow$ ${\tt;}$ \\ 

\item \noindent ${\it topology}$ $\rightarrow$ $({\it host\_group})^{*}$ $({\it role})^{*}$ \\ 
\item ${\it topology}$ $\rightarrow$ $({\it service\_group})^{*}$ $({\it role})^{*}$ \\ 
\item ${\it host\_groups}$ $\rightarrow$ $({\it host})^{+}$ $({\it dns})^{*}$ $({\it def\_gw})^{*}$ \\
\item ${\it dns}$ $\rightarrow$ ${\it host}$ \\ 
\item ${\it def\_gw}$ $\rightarrow$ ${\it host}$ \\ 
\item ${\it host}$ $\rightarrow$ (${\it interface\_name}$) (?) ${\it ip\_addr}$ ${\it netmask}$ \\ 
\item ${\it netmask}$ $\rightarrow$ ${\it ip\_addr}$ \\ 
\item ${\it interface\_name}$ $\rightarrow$ $[{\tt 0 - 9, a - z, A - Z}]^{+}$ \\ 
\item ${\it ip\_addr}$ $\rightarrow$ {\tt [0-255]}.{\tt [0-255]}.{\tt [0-255]}.{\tt [0-255]}\\
\item ${\it service\_group}$ $\rightarrow$ ${\it ip\_addr}$ ${\it netmask}$ ${\it serv\_listen\_port}$ \\ 
\item ${\it serv\_listen\_port}$ $\rightarrow$ {\tt [0-65536]}\\\\

\item \noindent ${\it role}$ $\rightarrow$ $({\it policy})^{*}$ \\
\item ${\it policy}$ $\rightarrow$ ${\it direction}$ ${\it verdict}$ ${proto}$ ${\it port}$ ${\it port}$ \\ 
\item ${\it policy}$ $\rightarrow$ ${\it direction}$ ${\it verdict}$ ${\it icmp\_cntrl\_message}$ \\ 
\item ${\it direction}$ $\rightarrow$ ${\tt inbound}$ \\ 
\item ${\it direction}$ $\rightarrow$ ${\tt outbound}$ \\ 
\item ${\it verdict}$ $\rightarrow$ $({\tt allow}|{\tt deny})$ \\ 
\item ${\it proto}$ $\rightarrow$ ({\tt UDP $ | $ TCP}) \\ 
\item ${\it icmp\_cntrl\_message}$ $\rightarrow$ ({\tt ICMP\_MESSAGE\_TYPE}) \\ 
\item ${\it port}$ $\rightarrow$ {\tt [0-65535]}\\
\end{enumerate}

  Clear seperation of routing information and firewalling policy description
  is evident from the seperation of rules for both both tasks.
   For the sake of testing, we choose the back - end for LANCOM as Linux
  {\it iptables} or FreeBSD {\it pfilter}. However, the flexibility of LANCOM 
   shall allow any administrator to specify configuration, independent of the 
   vendor specific syntax. A sample program in the next subsection shows how
  the language may be translated into Linux routing and firewalling commands.

 \subsection{Sample Program Snippet Using LANCOM}

  We next show a small example program written using LANCOM and a possible
  equivalent translation to Linux routing and 'iptables' firewalling
  configuration. \\

 \noindent \{\\
  \indent ${\tt topology\_t}$ $tp$;\\
  \indent ${\tt host\_group\_t}$ $hg$;\\
  \indent ${\tt service\_group\_t}$ $sg$;\\
  \indent ${\tt hg}$ ${\tt host\_group}$ ${\tt eth0}$ $128.59.0.0$ $255.255.0.0$ $128.59.16.20$ $255.255.255.255$
          $128.59.16.1$ $255.255.255.255$;\\

  \indent ${\tt role\_group\_t}$ $rg$;\\
  \indent ${\tt policy\_t}$ $pl$;\\
  \indent ${pl}$ ${\tt policy}$ ${\tt inbound}$ ${\tt TCP}$ ${\tt allow}$ $80$ ${\tt any}$;\\
  \indent ${rg}$ ${\tt role\_group}$ $pl$;\\
  \indent ${sg}$ ${\tt service\_group}$ $128.59.23.168$ $255.255.255.255$ $80$;\\
  \indent ${tp}$ ${\tt host\_group}$ $hg$;\\
  \indent ${tp}$ ${\tt service\_group}$ $sg$ $rg$;\\

\}


    The first line of the program begins by declaring an object of type ${\tt topology\_t}$.
    As per LANCOM grammar it is a collection of attributes which are objects of other types such 
    as ${\tt host\_group\_t}$. The attributes of ${\tt hg}$, ${\tt host\_group}$ is expanded from
    the rule $27$. It specifies the IP address of the subnet connected to network interface
    ${\tt eth0}$, the DNS server address and the default gateway. The ${\tt role\_group\_t}$ 
    object $rg$, is defined to contain policy $pl$. The policy $pl$ is to be translated to a 
    firewalling rule which should allow inbound connections to $TCP$ port $80$.The ${\tt service\_group}$
    object ${\tt sg}$ represents a web server waiting for incoming connection on IP 
    address $128.59.23.168$ and port $80$.

    The following is the equivalent routing configuration which results from
    compilation of the above program.\\ 

\noindent ${\tt route}$ ${\tt add}$ ${\tt -net}$ $128.59.0.0/16$ ${\tt dev}$ ${\tt eth0}$ \\

    All packets with destination IP address in subnet $128.59.0.0/16$ would be sent out of 
    network interface ${\tt eth0}$. For each host belonging to the subnet, the DNS server and
    default gateway would be $128.59.16.20$ and $128.59.16.1$ respectively. This information 
    may be used within the ${\tt DHCP} $ server process of the router and may be send to clients 
    through ${\tt DHCP}$ ${\tt offer}$ messages.

    The following are the equivalent Linux 'iptables' commands for the above
    rules. \\

 \noindent {\tt iptables -A FORWARD -d 128.59.0.0/16 --dest-port 80 -j ACCEPT}\\
           {\tt iptables - A FORWARD -d 128.59.23.168/32 --dest -port 80 -j ACCEPT} \\
          
          All this was simply a very rudimentary example demonstrating of how we intend administrators to use 
          LANCOM. The language described above is albeit may have ambiguous statements and parsing conflicts which we 
          plan to resolve prior to the implementation of our prototype.
