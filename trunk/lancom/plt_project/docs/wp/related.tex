\section{Related Work}
 \label{sec:related_work}

We next describe in brief, some research work relevant to our efforts.
  We being with describing a language called Firmato and conclude with the
  efforts presented by S.Ioannidis as a part of his doctoral thesis. 
  Firmato \cite{bartal99firmato} is perhaps the most relevant research in the area 
  of machine independent security policy description languages.  The objective of 
  Firmato is device independent security policy description.Secuity policies are 
  described in terms of objects with characteristics and properties.Further these 
  objects may be grouped into sets. These sets may be specified in the form of 
  `${\tt host\_group}$' and `${\tt service\_group}$'. A `${\tt host\_group}$ '
    is a set of hosts (say possibly within the same subnet. Fimato relies heavily 
  on attribute and property inheritance/seperation (for objects belong to the set contaning it). 
  This is achieved through a concept of $CLOSED$ and $OPEN$ addition to sets. When a new `${\tt host}$' 
  object is added to the group it inherits the properties of the set by default (addition to a set is $OPEN$ by default).
  However, when addition is explicity $CLOSED$, the object doesn 't inherit the security policies of the set.

Another research, closely relevant with our efforts, uses Lisp like language definition for 
defining packet filtering policies \cite{guttmanfiltering}. The language consists of notations
 for representing network specification, services and policies.  The authors however don't
 clarify how it can be used to generate firewall rules.

We conclude this section with a very relevant work done by S.Ioannidis \cite{ioannidis}.
The author addresses issues related to securing heterogenous and complex computer networks.
The dissertation discusses complexities arising from various network elements and the 
topologies through high level security and access control policy description.
The dissertaion highlights enforcing organizational level security policies
through automated and semi-automated tasks. 

Through LANCOM we try to provide an out of the box solution to the problem of seucrity policy and network topology 
description. An administrator of a complex network may use LANCOM to configure an array of routers and firewalls effortlessly.
